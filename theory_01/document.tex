\documentclass[a4paper]{article}
\usepackage[utf8]{inputenc}
\usepackage{amsmath}
\usepackage{amssymb}
\usepackage[ngerman]{babel}
\title{Computergrafik}
\author{Lisa Kassebaum - 199856 \\ Andreas Pfohl - 184670 \\ Martin Knoll - 193321 }
\date{}
\begin{document}
\maketitle
\section{}
\subsection{}
  \begin{align*}
    \left|v\right|
    =
    \left|
    \begin{pmatrix}
      1 \\ 4 \\ 3
    \end{pmatrix}
    \right|
    =
    \sqrt{1^2 + 4^2 + 3^2}
    = \sqrt{26} &= 5,1
    \\[1em]
    v\text{ normalisiert}: \quad \frac{1}{\sqrt{26}} = 0,2 \quad \frac{4}{\sqrt{26}} = 0,78 \quad \frac{3}{\sqrt{26}} &= 0,58
    \\[1em]
    \left|v\text{ normalisiert}\right|: \quad \sqrt{0,2^2 + 0,78^2 + 0,58^2} &= 1
  \end{align*}

  \begin{align*}
    \left|v\right|
    =
    \left|
    \begin{pmatrix}
      0 \\ 0 \\ 12
    \end{pmatrix}
    \right|
    =
    \sqrt{0^2 + 0^2 + 12^2}
    = \sqrt{144} &= 12
    \\[1em]
    v\text{ normalisiert}: \quad \frac{0}{\sqrt{12}} = 0 \quad \frac{0}{\sqrt{12}} = 0 \quad \frac{12}{\sqrt{12}} &= 1
    \\[1em]
    \left|v\text{ normalisiert}\right|: \quad \sqrt{0^2 + 0^2 + 1^2} &= 1
  \end{align*}

  \begin{align*}
    \left|v\right|
    =
    \left|
    \begin{pmatrix}
      -2 \\ 0 \\ 1
    \end{pmatrix}
    \right|
    =
    \sqrt{(-2)^2 + 0^2 + 1^2}
    = \sqrt{5} &= 2,24
    \\[1em]
    v\text{ normalisiert}: \quad \frac{-2}{\sqrt{5}} = -0,89 \quad \frac{0}{\sqrt{5}} = 0 \quad \frac{1}{\sqrt{5}} &= 0,45
    \\[1em]
    \left|v\text{ normalisiert}\right|: \quad \sqrt{(-0,89)^2 + 0^2 + 0,45^2} &= 1
  \end{align*}

\subsection{}
  \begin{align*}
    \begin{pmatrix}
      1 \\ 4 \\ 7
    \end{pmatrix}
    \times
    \begin{pmatrix}
      -2 \\ 0 \\ 3
    \end{pmatrix}
    &= 19
    \\[1em]
    \begin{pmatrix}
      -5 \\ 1 \\ 3
    \end{pmatrix}
    \times
    \begin{pmatrix}
      4 \\ -2 \\ 1
    \end{pmatrix}
    &= -19
    \\[1em]
    \begin{pmatrix}
      -5 \\ 1 \\ 3
    \end{pmatrix}
    \times
    \begin{pmatrix}
      4 \\ -2 \\ 1
    \end{pmatrix}
    &= 0
  \end{align*}
\subsection{}
  \begin{align*}
    \begin{pmatrix}
      5 \\ 3 \\ 0
    \end{pmatrix}
    \times
    \begin{pmatrix}
      -2 \\ 4 \\ 0
    \end{pmatrix}
    &=
    \begin{pmatrix}
      0 \\ 0 \\ 26
    \end{pmatrix}
    \\[1em]
    \begin{pmatrix}
      1 \\ 0 \\ 0
    \end{pmatrix}
    \times
    \begin{pmatrix}
      0 \\ 1 \\ 1
    \end{pmatrix}
    &=
    \begin{pmatrix}
      0 \\ -1 \\ 1
    \end{pmatrix}
  \end{align*}
  \subsection{}
    \begin{align*}
      c &= \vec{a} \times \vec{b}
      =
      \begin{pmatrix}
      a_1 \\ a_2 \\ a_3
      \end{pmatrix}
      \times
      \begin{pmatrix}
      b_1 \\ b_2 \\ b_3
      \end{pmatrix}
      =
      \begin{pmatrix}
      a_2 b_3 - a_3 b_2 \\ a_3 b_1 - a_1 b_3 \\ a_1 b_2 - a_2 b_1
      \end{pmatrix}
      \\[1em]
      c \cdot \vec{a}
      &=
      \begin{pmatrix}
      a_2 b_3 - a_3 b_2 \\ a_3 b_1 - a_1 b_3 \\ a_1 b_2 - a_2 b_1
      \end{pmatrix}
      \begin{pmatrix}
      a_1 \\ a_2 \\ a_3
      \end{pmatrix}
      \\[1em]
      &= (a_2 b_3 - a_3 b_1) (a_1) + (a_3 b_1 - a_1 b_3) (a_2) + (a_1 b_2 - a_2 b_1) (a_3)
      \\[1em]
      &= a_1 a_2 b_3 - a_1 a_3 b_2 + a_2 a_3 b_1 - a_1 a_2 b_3 + a_1 a_3 b_2 - a_2 a_3 b_1
      = 0
      \\[1em]
      c \cdot \vec{b}
      &=
      \begin{pmatrix}
      a_2 b_3 - a_3 b_2 \\ a_3 b_1 - a_1 b_3 \\ a_1 b_2 - a_2 b_1
      \end{pmatrix}
      \begin{pmatrix}
      b_1 \\ b_2 \\ b_3
      \end{pmatrix}
      \\[1em]
      &= (a_2 b_3 - a_3 b_1) (b_1) + (a_3 b_1 - a_1 b_3) (b_2) + (a_1 b_2 - a_2 b_1) (b_3)
      \\[1em]
      &= a_2 b_1 b_3 - a_3 b_1 b_2 + a_3 b_1 b_2 - a_1 b_2 b_3 + a_1 b_3 b_2 - a_2 b_1 b_3
      = 0
    \end{align*}
  \subsection{}
    \begin{align*}
      a =
      \begin{pmatrix}
        3 \\ 0 \\ 4
      \end{pmatrix}
      &\quad
      b =
      \begin{pmatrix}
        -4 \\ -3 \\ 0
      \end{pmatrix}
      \\[1em]
      \text{a)}\quad a^Tb = \|a\|\|b\|\cos{(a)}\rightarrow \alpha &= \cos^{-1} (\frac{a^Tb}{\|a\|\|b\|}) \\[1em]
      \|a\| &= 5 \\[1em]
      \|b\| &= 5 \\[1em]
      \cos{(a)} \rightarrow \cos^{-1}(\frac{-12}{25}) &= 118,69^\circ
      \\[1em]
      \text{b)}\quad \|a \times b\| = \|a\|\|b\|\sin{(a)}\rightarrow \alpha &= \sin^{-1} (\frac{\|a \times b\|}{\|a\|\|b\|}) \\[1em]
      \|a\| &= 5 \\[1em]
      \|b\| &= 5 \\[1em]
      \begin{pmatrix}
        3 \\ 0 \\ 4
      \end{pmatrix}
      \times
      \begin{pmatrix}
        -4 \\ -3 \\ 0
      \end{pmatrix}
      =
      \left|\begin{pmatrix}
        12 \\ 16 \\ -9
      \end{pmatrix}\right|
      &= \sqrt{481}
      \\[1em]
      \sin{(a)} \rightarrow \sin^{-1}(\frac{\sqrt{481}}{25}) &= 61,31^\circ
    \end{align*}
    \\[1em]
    Mit dem Kreuzprodukt lassen sich Winkel von $0^\circ$ bis $90^\circ$ berechnen, danach fällt die Sinuskurve und der Ergebniswinkel muss von $180^\circ$ abgezogen werden. Folglich ist das Ergebnis des Skalarprodukts, $118,69^\circ$, richtig.
  \subsection{}
    \begin{align*}
      a &=
      \begin{pmatrix}
        3 \\ 0
      \end{pmatrix}
      \quad
      b =
      \begin{pmatrix}
        \frac{3}{5} \\ \frac{4}{5}
      \end{pmatrix}
      \\[1em]
      a &= a_\parallel + a_\perp
      \\[1em]
      a_\parallel &= (a \circ e_b) \cdot e_b
      \\[1em]
      &= \left[
      \begin{pmatrix}
        3 \\ 0
      \end{pmatrix}
      \circ
      \begin{pmatrix}
        \frac{3}{5} \\ \frac{4}{5}
      \end{pmatrix}
      \right] \cdot
      \begin{pmatrix}
        \frac{3}{5} \\ \frac{4}{5}
      \end{pmatrix}
      =
      \begin{pmatrix}
        \frac{27}{25} \\ \frac{36}{25}
      \end{pmatrix}
      \\[1em]
      a_\perp &= a - a_\parallel
      \\[1em]
      &=
      \begin{pmatrix}
        3 \\ 0
      \end{pmatrix}
      -
      \begin{pmatrix}
        \frac{27}{25} \\ \frac{36}{25}
      \end{pmatrix}
      =
      \begin{pmatrix}
        \frac{48}{25} \\ -\frac{36}{25}
      \end{pmatrix}
    \end{align*}
\end{document}
